\documentclass[12pt, letterpaper]{article}

\usepackage{graphicx}
%\usepackage{parskip}% http://ctan.org/pkg/parskip
\usepackage{setspace}
\usepackage{amsfonts}
\usepackage{amsmath}

\setstretch{1.05}
\begin{document}
\noindent{\Large\textbf{M311S24 Problem Set 1.2} \textit{Franchi-Pereira, Philip}} \\

\paragraph{Problem 5.} If X and Y are finite and Y is non-empty then
so is \(F(X, Y)\), and \(|F(X, Y)| = |Y|^{|X|}\).

\paragraph{Definition 1} Let \(\epsilon: F(\{x\}, Y) \rightarrow Y\) by \(\epsilon(\{(\{x\},y)\} ) = y, y \in Y\), and define an inverse \(\gamma: Y \rightarrow F(\{x\}, Y), \gamma(y) = \{(x, y)\}\). It is clear these two are inverses, since \( (\gamma \circ \epsilon)(\{(a, b)\}) = \gamma(\epsilon(\{(a, b)\})) = \gamma(b) = \{(a, b)\ \) and \(\epsilon \circ \gamma(b) = \epsilon(\gamma(b)) = \epsilon(\{(a,b)\}) = b\) for \(a \in \{a\}\) and \(b \in Y\).

\paragraph{Definition 2} Let \(X = A \cup B\) with \(A \cap B = \emptyset\) , then we have the restriction map \(\mathcal{C}: F(X,Y) \rightarrow F(A,Y) \times F(B,Y)\) given by \(\mathcal{C}(f) = (f|_A, f|_B)\). \(\mathcal{C}\) is bijective, which will be shown by defining an inverse \(\mathcal{D}: F(A,Y) \times F(B,Y) \rightarrow F(X,Y)\) such that \(\mathcal{D}((f|_A, f|_B) = \{(a, f|_A(a)) : a \in A \} \cup \{(b, f|_B(b)) : b \in B \}\), which will be labeled \(\delta_f\) for the remainder of the proof. 
\\\\
 First we will show that \((\mathcal{D} \circ \mathcal{C})(f) = f\) for some \(f \in F(X, Y)\). 
Note that \((\mathcal{D} \circ \mathcal{C})(f) = \mathcal{D}(\mathcal{C}(f))) = \mathcal{D}((f|_A, f|_B)) = \delta_f\). To show that \(\delta_f = f\), we will show that they are subsets of eachother. For all \(x \in X, x\) is either in \(A\) or \(B\), since they are disjoint. If \(x \in A\), then \(\delta_f(x) = f|_A(x) = f(x)\) and if \( x \in B\), then \(\delta_f(x) = f|_B(x) = f(x)\) and so \(\delta_f \subseteq f\). Next, for all \(x \in X, f(x) = f|_A(x) = \delta_f(x)\) if \(x \in A\), and \( f(x) = f|_B(x) = \delta_f(x)\) if \(x \in B\). Therefore \(f \subseteq \delta_f\) and so \(f = \delta_f\). Note that if \(A\) and \(B\) were not disjoint, then for an element \(x \in A \cap B\),  \(f|_A(x) \) may or may not equal \(f|_B(x)\), and so \(F(X,Y)\) may not be well ordered.
\\\\
Next we will show that \((\mathcal{C} \circ \mathcal{D})((f|_A, f|_B)) = (f|_A, f|_B)\). Note that  \((\mathcal{C} \circ \mathcal{D})((f|_A, f|_B)) =\mathcal{C}(\mathcal{D}((f|_A, f|_B))) =\mathcal{C}(\delta_f) = (\delta_f|_A, \delta_f|_B)\). To show that \((\delta_f|_A, \delta_f|_B) = (f|_A, f|_B)\), we must show that  \(\delta_f|_A = f|_A\) and \(\delta_f|_B = f|_B\). However, by definition of \(\delta_f\), for all \(a \in A, \delta_f(a) = f|_A\) and for all \(b \in b, \delta_f(b) = f|_B\). Therefore \((\delta_f|_A, \delta_f|_B) = (f|_A, f|_B)\),  \((\mathcal{C} \circ \mathcal{D})((f|_A, f|_B)) = (f|_A, f|_B)\), and so \(\mathcal{C}\) and \(\mathcal{D}\) are inverses.
\\\\
Finally, we will use induction to prove that for finite sets \(X\) and \(Y\) with \(Y \neq \emptyset\), then \(|F(X,Y)| = |Y|^{|X|}\).

\paragraph{Base Case} Let \(|X| = 1, X = \{x\}\). Since there exists a bijection \(\epsilon(F(\{x\}, Y) = Y\), \(|F(\{x\}, Y| = |Y|\), and since \(|Y|^{|X|} = |Y|^{|1|}  = |Y|\), \(|F(\{x\}, Y| = |Y|^{|X|} = |Y|\).

\paragraph{Inductive Proposition} Let \(A\) and \(Y\) be finite sets, with \(Y \neq \emptyset\) and \(|A| =n\).  Assume \(|A|= n\) implies \(|F(A, Y)| = |Y|^{|A|} =  |Y|^n\),  Then for some set \(X\) with \(|X| = n +1\), \(|F(X, Y)| = |Y|^{|X|} = |Y|^{n+1}\).

\paragraph{Proof}  Let \(A\) and \(Y\) be finite sets, with \(Y \neq \emptyset\) and \(|A| =n\). Let \(A = X - \{x\}\), and \(B = \{x\}\). It is clear that \(A \cap B = \emptyset\), and that \(X = A \cup B\).  Then by Definition 2 there exists a map \(\mathcal{C}: F(X,Y) \rightarrow F(A,Y) \times F(B,Y)\) and its inverse \(\mathcal{D}\). Since \(\mathcal{C}\) has an inverse \(\mathcal{D}\), it is a bijection and therefore \(|F(X,Y)| = |F(A,Y) \times F(B,Y)|\). By Corollary 2.2.17 in the class notes, \(|F(A,Y) \times F(B,Y)| = |F(A,Y)| \cdot |F(B,Y)| = |F(X,Y)|\). 
\\
Since \(|B| = 1, |F(B, Y)| = |Y|\), and by the inductive hypothesis \(|A| = n, |F(A,Y)| = |Y|^{n}\), then \(|F(X, Y)| = |Y|^n \cdot  |Y|^1 = |Y|^{|n+1|}\). Therefore, \(|F(X, Y)| = |Y|^{|n + 1|} = |Y|^{|X|}\).
 
\paragraph{Problem 6.} An Alternate Proof of \(|\mathcal{P}(X)| = 2^{|X|}\).
\\
\\
For a positive integer \(n\) we denote the set \(\{0, 1, 2, . . .n - 1\}\) by \(\mathbb{Z}_{n}\). Thus
\(\mathbb{Z}_2 = \{0, 1\}\) We have a map \(\Sigma : F(X,\mathbb{Z}_2) \rightarrow \mathcal{P}(X) , f \mapsto f^{-1}(1)\). Given \(A \subseteq X\) we define
\( \chi_{A} \in F(X,\mathbb{Z}_2)\) by \(\chi_A(x) = 1\) for \(x \in A\) and \(\chi_A(x) = 0\)  for \(x \notin A\). Last, define the map \(\Xi: (\mathcal{P}(X) \rightarrow F(X, \mathbb{Z}_2)\) by \(\Xi(A) =  \mathcal{X}_A\). 
\\
\\
First we will show that \(\Xi\) and \(\Sigma\) are inverses. Consider first \((\Sigma \circ \Xi)(A), A \subseteq X\). \( (\Sigma \circ \Xi)(A) = \Sigma(\Xi (A)) =\Sigma(\chi_A) =  \chi^{-1}_{A}(1)\). To show that \(\chi^{-1}_{A}(1) = A\), we will show that \(\chi^{-1}_{A}(1) \subseteq A\) and then \(A \subseteq \chi^{-1}_{A}(1)\). First, for all \(x \in \chi^{-1}_{A}(1), \chi_A(x) = 1,\) and so \(x \in A\), therefore \(\chi^{-1}_{A}(1) \subseteq A\). Next, for all \(a \in A, \chi_A(a) = 1\), and so \(a \in \chi^{-1}_{A}(1)\). Therefore, \(A \subseteq \chi^{-1}_{A}(1)\) and so \( A = \chi^{-1}_{A}(1)\) and  \((\Sigma \circ \Xi)(A) = A\).
\\
\\
Next, consider \((\Xi \circ \Sigma)(f) = \Xi(\Sigma(f)) = \Xi(f^{-1}(1)) = \chi_{ f^{-1}(1)}\). First we show that \(f \subseteq \chi_{ f^{-1}}(1)\). For all \(x \in X\), if \((x, 1) \in f, x \in f^{-1}(1)\), and so \((x, 1) \in \chi_{ f^{-1}(1)}\). In the case instead where \((x, 0) \in f\) the \(x \notin f^{-1}(1)\) and so \((x, 0) \notin \chi_{ f^{-1}(1)}\). Therefore \(f \subseteq \chi_{ f^{-1}(1)}\). To show that  \(\chi_{ f^{-1}(1)} \subseteq f\),  note that for all \( x \in X\), if \( \chi_{ f^{-1}(1)}(x) = 1\), then \(x \in f^{-1}(1)\) and therefore \((x, 0) \in f\). If \( \chi_{ f^{-1}(1)}(x) = 0\), then \(x \notin f^{-1}(1)\) and so \((x, 0) \in f\). Therefore  \(f \subseteq \chi_{ f^{-1}(1)}\), so \(f = \chi_{ f^{-1}(1)}, (\Xi \circ \Sigma)(f) = f\), and so \(\Xi\) and \(\Sigma\) are inverses.
\\
\\
Finally, since \(\Xi\) and \(\Sigma\) are inverses, then \(\Sigma\) is bijective, and so \(|\mathcal{P}(X)| = |F(X, \mathbb{Z}_{2})|\). By the proof demonstrated in Problem 5, \(|F(X, \mathbb{Z}_{2})| = |\mathbb{Z}_{2}|^{|X|}.\) Since \(\mathbb{Z}_2\) is known to only have the elements \(\{0, 1\}\), \(|\mathbb{Z}_2| = 2\), and so \(|F(X, \mathbb{Z}_{2})| = 2^{|X|}\). Therefore, \(|\mathcal{P}(X)| = |F(X, \mathbb{Z}_{2})| = 2^{|X|}\).


\paragraph{Problem 7.} \(\#(Gl(n, \mathbb{Z}_p)) = (p^n - 1)(p^n - p)(p^n - p^{2})...(p^n - p^{n-1}) =  p^{n^2}(1 - \frac{1}{p^n})(1 - \frac{1}{p^{n-1}})(1 - \frac{1}{p})\).
\\
\\
Since the size of the group \(Gl(n, \mathbb{Z}_p)\) of invertible \(n \times n\) matrices is the number of the possible ordered basis of \(\mathbb{Z}_p^n\), we will show that the number of orderings of a span of \(k\) vectors in \(\mathbb{Z}_p^n = (p^n - 1)(p^n - p)(p^n - p^{2})...(p^n - p^{k-1})\). We will induce on \(k\), limiting the induction such that \(k \leq n\), since if the span contains more than \(n\) vectors they are by definition no longer independent.

\paragraph{Base Case} \(k = 1\) \\
Each vector has \(n\) components, each with \(p\) possible values. The choice of \(v_k\) could be any of the \(p^n\) possible vectors, except for the 0 vector. So there are \(p^n - 1\) choices for \(v_0\). The number of orderings of the span are therefore \(p^n - p^{k-1} = p^n - p^{1-1} = p^n - 1\).

\paragraph{Base Case} \(k = 2\) \\
There are \(p^n\) possible choices for \(v_2\), but it cannot be a scalar multiple of  
\(span(v_1)\). Since there is only one vector in the \(span(v_1)\), then there are \(p\) scalar mutlipes of existing vectors in the span that \(v_2\) cannot be chosen, in order to maintain independence. There are then \((p^n - p)\) choices for \(v_2\), and therefore \((p^n - p)(p^n - 1)\) possible orderings of the span.
\\\\
\textbf{Inductive Proposition}.
\\
Assume the number of orderings of a \(span(v_1, v_2, . . . , v_k) = (p^n - 1)(p^n - p)(p^n - p^{2})...(p^n - p^{k-1})\). Then for any number \(a \in \mathbb{N}, a \leq n, a = k + 1\), the number of orderings of vectors in \(span(v_1, v_2, . . . , v_a) = (p^n - 1)(p^n - p)(p^n - p^{2})...(p^n - p^{a-1})\).
\\
Again, the choice of the next vector \(v_a\) could be one of any \(p^n\) vectors. However, we cannot pick any scalar multiple of a vector already in \(span(v_1, v_2, . . ., v_{a-1})\). There are \(a - 1\) vectors in the span, and so there are \(p^{a-1}\) vectors that cannot be chosen, which makes the number of choices for \(v_a = (p^n - p^{a-1}= p^n - p^k\), since \(a = k +1, k = a - 1\). By the inductive hypothesis, the total number of orderings of the previous \{a-1\} vectors, \(span(v_1,v_2, . . ., v_k)\) is assumed to be \((p^n - 1)(p^n - p)(p^n - p^{2})...(p^n - p^{k-1})\). Therefore the orderings of the span, including \(v_a\) are 
\begin{align*}
(p^{n} - p^k ) \times (p^n - 1)(p^n - p)(p^n - p^{2})...(p^n - p^{k-1}) =\\
 (p^n - p)(p^n - p^{2})...(p^n - p^{k-1})(p^n - p^k) =\\
  (p^n - p)(p^n - p^{2})...(p^n - p^{a-2})(p^n - p^{a-1}). \end{align*}
which proves the inductive proposition.
\\
\\
Finally, since the size of \(Gl(n, \mathbb{Z}_p))\) is equal to the number of orderings of basis in \(\mathbb{Z}_p^n\), and there are \(n\) vectors in the span, then \(\#(Gl(n, \mathbb{Z}_p)) = (p^n - 1)(p^n - p)(p^n - p^{2})...(p^n - p^{n-1})\).
\end{document}