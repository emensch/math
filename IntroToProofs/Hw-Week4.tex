\documentclass[12pt, oneside]{article}
\usepackage{amsmath}
\usepackage{amsfonts}
\usepackage[parfill]{parskip}
\usepackage{hyperref}
\newcommand{\Reals}{\mathbb{R}}
\newcommand{\Integers}{\mathbb{Z}}
\newcommand{\Naturals}{\mathbb{N}}
\newcommand{\Rationals}{\mathbb{Q}}
\newcommand{\aandb}{\(a\) and \(b\)}

\begin{document}
\begin{flushleft}
    \noindent{\Large\textbf{Math 156, Sec 4, H.W. 4} \textit{Franchi-Pereira, Philip}}
\end{flushleft}
February, 26, 2024\\\\
\begin{itemize}

    \item[Problem 1] Suppose \(n \in Integers\). If \(n^2\) is odd then \(n\) is odd.

          \textbf{Proof by contrapositive}: Suppose \(n\) is even, then there exists a \(k \in \Integers\) such that \(n = 2k\). Therefore, \(n^2 = {(2k)}^2 = 4k^2 = 2(2k^2)\), and since \(2k^2\) is also an integer, \(2(2k^2)\) is even, and so \(n^2\) is even.

    \item[Problem 2] Suppose \(a, b, c \in Integers\). If \(a\) does not divide \(bc\) then \(a\) does not divide \(b\).

          \textbf{Proof by Contrapositive}: Suppose \(a\) does divide \(b\). Then there exists a \(q \in \Integers\) such that \(b = aq\), and so \(bc = aqc\). It is clear that \(a|aqc\). Therefore \(a|bc\).

    \item[Problem 3] For any \(a, b \in \Integers\), it follows that \({(a + b)}^{3} = a^3 + b^3\) (mod 3).

          \textbf{Direct Proof}: Expanding the terms, we see that \({(a + b)}^{3} = (a^3 + b^3) + 3(a^2b+ab^2)\), and so \({(a + b)}^{3} - (a^3 + b^3) = (a^3 + b^3) + 3(a^2b+ab^2) - (a^3 + b^3) = 3(a^2b+ab^2)\). Since \(3|{(a + b)}^{3} - (a^3 + b^3)\), by the definition of modulus it follows that \({(a + b)}^{3} = a^3 + b^3\) (mod 3).

    \item[Problem 4] Suppose \(x \in \Reals\). If \(x^3 - x > 0\), then \(x > -1\).

          \textbf{Proof by Contrapositive} Suppose instead \(x \leq -1\). Then \(x^2 \geq 1\). Next multiply both sides by \(x\), and since \(x\) is known to be negative, we flip the inequality to get \(x^3 \leq x\) and finally \(x^3 - x \leq 0\).

    \item[Note for the HW.] For the following proofs, the logic statement required by the assignment is the first line of the proof.

    \item[Problem 5] There exist no integers \aandb{} for which \(21a + 30b = 1\).
          \textbf{Proof by contradiction}: Suppose there exists two integers \aandb{} such that \(21a + 30b = 1\). That would mean that \(3(7a + 10b) = 1\). Letting \(k = 7a + 10b\), the equation becomes \(3k = 1\). This is a contradiction, as there is no such integer.

    \item[Problem 6] If \aandb{} are positive real numbers, then \(a + b \geq 2\sqrt{ab}\).

          \textbf{Proof by contradiction}: Suppose instead that if \aandb{} are positive real numbers, then \(a + b < 2\sqrt{ab}\). Without loss of generality, say \(a \geq b\). Then \(a = b + r\), for some \( r \in \Reals\) and so \(a + b = 2b + r\). Meanwhile \(2\sqrt{ab} = 2\sqrt{b^2 + br}\). Since both sides of the inequality are clearly positive, we may square both sides, showing
          \begin{equation*}
              \begin{split}
                  {(2b +r)}^2     \  & <\ 2\sqrt{b^2 + br} \\
                  4b^2 + 4br + r^2\  & <\ 4b^2 + 4br       \\
                  r^2\               & <\ 0
              \end{split}
          \end{equation*}
          However, for any value of \(r \in \Reals,\ r^2\) is clearly greater than or equal to 0 and so this is a contradiction.

    \item[Problem 7] Prove by contradiction that no odd integer can be expressed as the sum of three.
          even integers.

          \textbf{Proof by contradiction}: Assume there exists an odd integer \(x\), and even integers \(a, b, c\) such that \(x = a + b +c\). Since \(a, b,c\) are even, there exists some \(k_1, k_2, k_3 \in \Integers\) such that \(a = 2k_1,\ b=2k_2,\ c=2k_3\). The sum of these integers is then \(x = a + b + c = 2k_1 + 2k_2 + 2k_3 = 2(k_1 + k_2 + k_3)\). Since \(k_1+ k_2+ k_3\) is an integer, \(2(k_1+ k_2+ k_3)\) is even. But since \(x\) was assumed to be odd, that is a contradiction.
\end{itemize}
\end{document}